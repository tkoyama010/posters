\documentclass[final]{beamer}
\mode<presentation> {
  \usetheme{metropolis}
}

\usepackage[orientation=landscape,size=a0]{beamerposter}
\usepackage{lipsum}
% \usepackage[colorgrid,gridunit=pt,texcoord]{eso-pic}
\usepackage[absolute,overlay]{textpos}
\usepackage{pythonhighlight}
\usepackage[absolute,overlay]{textpos}
\usepackage{url}
\usepackage{caption}
\usepackage{hyperref}
\usepackage[
    type={CC},
    modifier={by},
    version={3.0},
]{doclicense}

\lstset{
language = Python,
breaklines = true,
basicstyle=\fontsize{18}{20}\selectfont\ttfamily,
commentstyle = {\itshape \color[cmyk]{1,0.4,1,0}},
keywordstyle = {\bfseries \color[cmyk]{0,1,0,0}},
stringstyle = {\ttfamily \color[rgb]{1,0,0}},
frame = single,
}
\hypersetup{
colorlinks=true,
}

\begin{document}
\begin{frame}[fragile]
\begin{textblock*}{3450pt}(50pt, 50pt)
\Huge Visualize 3D scientific data in a Pythonic way like matplotlib

\Large Tetsuo Koyama
\end{textblock*}

\begin{textblock*}{800pt}(50pt, 200pt)
\begin{block}{Abstract}
Do you want to visualize 3D scientific data in a Pythonic way like matplotlib?
If you want, this poster is for you.
This poster is the introduction of \href{https://pypi.org/project/pyvista/}{PyVista}.
It is
\begin{itemize}
\item "VTK for humans"\: a high-level API to the Visualization Toolkit (VTK)
\item 3D plotting made simple and built for large/complex data geometries
\item mesh data structures and filtering methods for spatial datasets
\end{itemize}

\end{block}
\begin{block}{Hello World!}
In code \ref{hello_world_code}, we demonstrate the "Hello World!" of PyVista.
Basic step of PyVista script is the following.
First, import \href{https://pypi.org/project/pyvista/}{PyVista}.
Then generate \href{https://docs.pyvista.org/getting-started/what-is-a-mesh.html}{Mesh} and add it to
\href{https://docs.pyvista.org/plotting/plotting.html#pyvista.Plotter}{Plotter} object using \href{https://docs.pyvista.org/plotting/plotting.html#pyvista.BasePlotter.add\_mesh}{add\_mesh} method.
And finally, we can check the render view (Figure \ref{HelloWorldFigure}) of PyVista using \href{https://docs.pyvista.org/plotting/plotting.html#pyvista.Plotter.show}{\texttt{show}} method.

\lstinputlisting[caption=Hello World!, label=hello_world_code, firstline=1, lastline=100]{hello_world.py}
\begin{figure}
\includegraphics[width=1.0\linewidth]{hello_world.png}
\caption{Hello World!}\label{HelloWorldFigure}
\end{figure}
\end{block}
\begin{block}{General filters to any data type}
These classes hold methods to apply general filters to any data type.
By inheriting these classes into the wrapped VTK data structures, a user can easily apply common filters in an intuitive manner.
\lstinputlisting[caption=Shrink Mesh, label=shrunk_mesh, firstline=4, lastline=5]{shrunk_mesh.py}
\begin{figure}
\includegraphics[width=1.0\linewidth]{shrink.png}
\caption{Shrink filter}
\end{figure}
\end{block}
\begin{block}{Extrude Rotation}
AAAAAAAAAAA
\lstinputlisting[caption=Extrude Rotate, label=extrude_rotate, firstline=7, lastline=9]{extrude_rotate.py}
\begin{figure}
\includegraphics[width=1.0\linewidth]{extrude_rotate.png}
\caption{Extrude Rotation}
\end{figure}
\end{block}
\end{textblock*}

\begin{textblock*}{800pt}(900pt, 200pt)
\begin{block}{Load and Plot from a File}
The same stl can be loaded and plotted using PyVista with:
The mesh object is more pythonic and the code is much more straightforward.
Garbage collection is taken care of automatically and the renderer is cleaned up after the user closes the VTK plotting window.
\lstinputlisting[caption=Load stanford bunny from file, label=read_file, firstline=6, lastline=6]{read_file.py}
\begin{figure}
\includegraphics[width=1.0\linewidth]{read_file.png}
\caption{Stanford bunny (Load from file)}
\end{figure}
\end{block}
\end{textblock*}

\begin{textblock*}{800pt}(1750pt, 50pt)
\begin{block}{Camera class}
Camera class is a virtual camera for 3D rendering.
It provides methods to position and orient the view point and focal point.
Convenience methods for moving about the focal point also are provided.
More complex methods allow the manipulation of the computer graphics model including view up vector, clipping planes, and camera perspective.

The pyvista.Camera class adds additional functionality and a pythonic API to the vtk.vtkCamera class.
pyvista.Camera objects come with a default set of cameras that work well in most cases,
but in many situations a more hands-on access to camera is necessary.

Create a frustum of camera, then create a scene of inside frustum.

\begin{figure}
\includegraphics[width=1.0\linewidth]{frustum_of_camera.png}
\caption{frustum of camera}
\end{figure}
\end{block}
\end{textblock*}

\begin{textblock*}{800pt}(2570pt, 50pt)
\begin{block}{Controlling Camera Rotation}
In addition to directly controlling the camera position by setting it via the pyvista.
Camera.position property, you can also directly control the pyvista.Camera.roll, pyvista.Camera.elevation, and pyvista.Camera.azimuth of the camera.
\lstinputlisting[caption=Camera View, label=camera_view, firstline=7, lastline=13]{camera_view.py}
\begin{figure}
\includegraphics[width=1.0\linewidth]{camera_view.png}
\caption{Camera View}
\end{figure}
\end{block}
\end{textblock*}

\begin{textblock*}{800pt}(900pt, 1200pt)
\begin{block}{Contouring}
AAAAAAAAAAA
\lstinputlisting[caption=Plotting contour, label=Contour, firstline=12, lastline=21]{contour.py}
\begin{figure}
\includegraphics[width=1.0\linewidth]{contour.png}
\caption{Contouring}
\end{figure}
\end{block}
\end{textblock*}

\begin{textblock*}{800pt}(1750pt, 1200pt)
\begin{block}{Plot data over circular arc}
Plot data over circular arc
\begin{figure}
\includegraphics[width=1.0\linewidth]{kitchen.png}
\caption{kitchen}
\end{figure}
\end{block}
\end{textblock*}

\begin{textblock*}{800pt}(2570pt, 1200pt)
\lstinputlisting[caption=Plotting over line, label=kitchen, firstline=11, lastline=18]{kitchen.py}
\begin{figure}
\includegraphics[width=1.0\linewidth]{velocity.png}
\caption{Plot over line}
\end{figure}
\end{textblock*}

\begin{textblock*}{800pt}(2570pt, 2100pt)
\begin{block}{References}
Sullivan et al., (2019). PyVista: 3D plotting and mesh analysis through a streamlined interface for the Visualization Toolkit (VTK). Journal of Open Source Software, 4(37), 1450, \url{https://doi.org/10.21105/joss.01450}
\end{block}
\end{textblock*}

\begin{textblock*}{800pt}(2570pt, 2250pt)
\begin{block}{Contact Information}
\end{block}
\doclicenseThis
\end{textblock*}

\end{frame}
\end{document}

