\documentclass[final]{beamer}
\mode<presentation> {
  \usetheme{metropolis}
}

\usepackage[orientation=landscape,size=a0]{beamerposter}
\usepackage{lipsum}
\usepackage[colorgrid,gridunit=pt,texcoord]{eso-pic}
\usepackage[absolute,overlay]{textpos}
\usepackage{pythonhighlight}
\usepackage[absolute,overlay]{textpos}

\begin{document}
\begin{textblock*}{3450pt}(50pt, 50pt)
\Huge Visualize 3D scientific data in a Pythonic way like matplotlib

\Large Tetsuo Koyama
\end{textblock*}

\begin{textblock*}{800pt}(50pt, 200pt)
\begin{block}{Abstract}
Abstract
\end{block}
\end{textblock*}

\begin{textblock*}{800pt}(50pt, 350pt)
\begin{block}{Pythonic interface to 3D visualization}
The same stl can be loaded and plotted using pyvista with:
The mesh object is more pythonic and the code is much more straightforward.
Garbage collection is taken care of automatically and the renderer is cleaned up after the user closes the VTK plotting window.
\inputpython{hello_world.py}{1}{100}
\begin{figure}
\includegraphics[width=1.0\linewidth]{hello_world.png}
\caption{Hello World}
\end{figure}
\end{block}
\end{textblock*}

\begin{textblock*}{800pt}(900pt, 200pt)
\begin{block}{Load and Plot from a File}
AAAAA
\inputpython{read_file.py}{1}{100}
\begin{figure}
\includegraphics[width=1.0\linewidth]{read_file.png}
\caption{Hello World}
\end{figure}
\end{block}
\end{textblock*}

\begin{textblock*}{800pt}(50pt, 1350pt)
\begin{block}{General filters to any data type}
These classes hold methods to apply general filters to any data type.
By inheriting these classes into the wrapped VTK data structures, a user can easily apply common filters in an intuitive manner.
\inputpython{shrunk_mesh.py}{4}{5}
\begin{figure}
\includegraphics[width=1.0\linewidth]{shrink.png}
\caption{Shrink filter}
\end{figure}
\end{block}
\begin{block}{Extrude Rotation}
AAAAAAAAAAA
\inputpython{extrude_rotate.py}{7}{9}
\begin{figure}
\includegraphics[width=1.0\linewidth]{extrude_rotate.png}
\caption{Extrude Rotation}
\end{figure}
\end{block}
\end{textblock*}

\begin{textblock*}{800pt}(1750pt, 50pt)
\begin{block}{Camera class}
The pyvista.Camera class adds additional functionality and a pythonic API to the vtk.vtkCamera class.
pyvista.Camera objects come with a default set of cameras that work well in most cases,
but in many situations a more hands-on access to camera is necessary.

Create a frustum of camera, then create a scene of inside frustum.

\begin{figure}
\includegraphics[width=1.0\linewidth]{frustum_of_camera.png}
\caption{frustum of camera}
\end{figure}
\end{block}
\end{textblock*}

\begin{textblock*}{800pt}(2570pt, 50pt)
\inputpython{camera_view.py}{1}{100}
\begin{figure}
\includegraphics[width=1.0\linewidth]{camera_view.png}
\caption{Camera View}
\end{figure}
\end{textblock*}

\begin{textblock*}{800pt}(900pt, 1200pt)
\begin{block}{Contouring}
AAAAAAAAAAA
\inputpython{contour.py}{13}{19}
\includegraphics[width=1.0\linewidth]{contour.png}
\end{block}
\end{textblock*}

\begin{textblock*}{800pt}(1750pt, 1200pt)
\begin{block}{Plot data over circular arc}
Plot data over circular arc
\begin{figure}
\includegraphics[width=1.0\linewidth]{kitchen.png}
\caption{kitchen}
\end{figure}
\end{block}
\end{textblock*}

\begin{textblock*}{800pt}(2570pt, 1200pt)
\inputpython{kitchen.py}{11}{18}
\begin{figure}
\includegraphics[width=1.0\linewidth]{velocity.png}
\caption{Plot over line}
\end{figure}
\end{textblock*}

\begin{textblock*}{800pt}(2570pt, 2150pt)
\begin{block}{References}
\end{block}
\end{textblock*}

\begin{textblock*}{800pt}(2570pt, 2250pt)
\begin{block}{Contact Information}
\end{block}
\end{textblock*}

\end{document}

