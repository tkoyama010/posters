%%%%%%%%%%%%%%%%%%%%%%%%%%%%%%%%%%%%%%%%%
% Jacobs Landscape Poster
% LaTeX Template
% Version 1.0 (29/03/13)
%
% Created by:
% Computational Physics and Biophysics Group, Jacobs University
% https://teamwork.jacobs-university.de:8443/confluence/display/CoPandBiG/LaTeX+Poster
% 
% Further modified by:
% Nathaniel Johnston (nathaniel@njohnston.ca)
%
% This template has been downloaded from:
% http://www.LaTeXTemplates.com
%
% License:
% CC BY-NC-SA 3.0 (http://creativecommons.org/licenses/by-nc-sa/3.0/)
%
%%%%%%%%%%%%%%%%%%%%%%%%%%%%%%%%%%%%%%%%%

%----------------------------------------------------------------------------------------
%	PACKAGES AND OTHER DOCUMENT CONFIGURATIONS
%----------------------------------------------------------------------------------------

\documentclass[final]{beamer}

\usepackage[scale=1.24]{beamerposter} % Use the beamerposter package for laying out the poster
\usepackage[colorgrid,gridunit=pt,texcoord]{eso-pic}
\usepackage[absolute,overlay]{textpos}

\usetheme{confposter} % Use the confposter theme supplied with this template

\setbeamercolor{block title}{fg=ngreen,bg=white} % Colors of the block titles
\setbeamercolor{block body}{fg=black,bg=white} % Colors of the body of blocks
\setbeamercolor{block alerted title}{fg=white,bg=dblue!70} % Colors of the highlighted block titles
\setbeamercolor{block alerted body}{fg=black,bg=dblue!10} % Colors of the body of highlighted blocks
% Many more colors are available for use in beamerthemeconfposter.sty

%-----------------------------------------------------------
% Define the column widths and overall poster size
% To set effective sepwid, onecolwid and twocolwid values, first choose how many columns you want and how much separation you want between columns
% In this template, the separation width chosen is 0.024 of the paper width and a 4-column layout
% onecolwid should therefore be (1-(# of columns+1)*sepwid)/# of columns e.g. (1-(4+1)*0.024)/4 = 0.22
% Set twocolwid to be (2*onecolwid)+sepwid = 0.464
% Set threecolwid to be (3*onecolwid)+2*sepwid = 0.708

\newlength{\sepwid}
\newlength{\onecolwid}
\newlength{\twocolwid}
\newlength{\threecolwid}
\setlength{\paperwidth}{48in} % A0 width: 46.8in
\setlength{\paperheight}{36in} % A0 height: 33.1in
\setlength{\sepwid}{0.024\paperwidth} % Separation width (white space) between columns
\setlength{\onecolwid}{0.22\paperwidth} % Width of one column
\setlength{\twocolwid}{0.464\paperwidth} % Width of two columns
\setlength{\threecolwid}{0.708\paperwidth} % Width of three columns
\setlength{\topmargin}{-0.5in} % Reduce the top margin size
%-----------------------------------------------------------

\usepackage{graphicx}  % Required for including images

\usepackage{booktabs} % Top and bottom rules for tables

\usepackage{pythonhighlight}
%----------------------------------------------------------------------------------------
%	TITLE SECTION 
%----------------------------------------------------------------------------------------

\title{Visualize 3D scientific data in a Pythonic way like matplotlib} % Poster title

\author{Tetsuo Koyama} % Author(s)

\institute{PyVista developers team} % Institution(s)

%----------------------------------------------------------------------------------------

\begin{document}

\addtobeamertemplate{block end}{}{\vspace*{2ex}} % White space under blocks
\addtobeamertemplate{block alerted end}{}{\vspace*{2ex}} % White space under highlighted (alert) blocks

\setlength{\belowcaptionskip}{2ex} % White space under figures
\setlength\belowdisplayshortskip{2ex} % White space under equations

\begin{textblock*}{800pt}(100pt, 450pt)
\begin{alertblock}{Abstract}
\end{alertblock}
\end{textblock*}

\begin{textblock*}{800pt}(100pt, 700pt)
\begin{block}{Pythonic interface to 3D visualization}
The same stl can be loaded and plotted using pyvista with:
The mesh object is more pythonic and the code is much more straightforward.
Garbage collection is taken care of automatically and the renderer is cleaned up after the user closes the VTK plotting window.
\inputpython{hello_world.py}{1}{100}
\begin{figure}
\includegraphics[width=1.0\linewidth]{hello_world.png}
\caption{Hello World}
\end{figure}
\end{block}
\end{textblock*}

\begin{textblock*}{800pt}(950pt, 450pt)
\begin{block}{General filters to any data type}
These classes hold methods to apply general filters to any data type.
By inheriting these classes into the wrapped VTK data structures, a user can easily apply common filters in an intuitive manner.
\inputpython{shrunk_mesh.py}{1}{100}
\begin{figure}
\includegraphics[width=1.0\linewidth]{shrink.png}
\caption{Shrink filter}
\end{figure}
\end{block}
\end{textblock*}

\begin{textblock*}{800pt}(1800pt, 450pt)
\begin{block}{About Camera object}
The pyvista.Camera class adds additional functionality and a pythonic API to the vtk.vtkCamera class.
pyvista.Camera objects come with a default set of cameras that work well in most cases,
but in many situations a more hands-on access to camera is necessary.

Create a frustum of camera, then create a scene of inside frustum.

\begin{figure}
\includegraphics[width=1.0\linewidth]{frustum_of_camera.png}
\caption{frustum of camera}
\end{figure}

\inputpython{camera_view.py}{1}{100}
\end{block}
\end{textblock*}

\begin{textblock*}{800pt}(2650pt, 700pt)
\begin{block}{Plot data over line and circular arc}
\end{block}
\end{textblock*}

\begin{textblock*}{800pt}(2650pt, 450pt)
\begin{block}{References}
\end{block}
\end{textblock*}

\setbeamercolor{block alerted title}{fg=black,bg=norange} % Change the alert block title colors
\setbeamercolor{block alerted body}{fg=black,bg=white} % Change the alert block body colors

\begin{textblock*}{800pt}(2650pt, 900pt)
\begin{alertblock}{Contact Information}
\end{alertblock}
\end{textblock*}

\end{document}

